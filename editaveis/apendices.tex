\begin{apendicesenv}
\partapendices
\chapter[Revisão Sistemática]{Revisão Sistemática}
Para a concepção e definição de um protocolo de revisão que visa, especificar a pesquisa, as questões e os métodos utilizados, foi produzido neste apêndice um resumo das etapas de uma revisão sistemática definidas em três fases principais por \cite{kitchenham2004procedures} que serão aplicadas na execução deste estudo secundário: 

\hypertarget{A}{}
\section{\textbf{Planejamento da revisão:}}
As etapas associadas ao planejamento da revisão são:

\subsection{Desenvolvimento de um protocolo de revisão.}
Foi desenvolvido um protocolo de revisão, o qual possibilita, a outros pesquisadores, a repetição da pesquisa. O mesmo é denotado abaixo.

\subsubsection{Questões de Pesquisa}
Para definição das questões utilizadas para pesquisa foi utilizado como critério a relação com a tema deliberado preliminarmente. Dessa forma foi acordado a existência de perguntas de acordo com a base utilizada apenas nos idiomas: 

- Português

- Inglês 

\begin{itemize}
    \item Existem trabalhos que correlatam as práticas das equipes ágeis no processo de elicitação. modelagem e análise de requisítos? 

    \item O que é feito nos níveis mais altos de abstração?

    \item Quanto tempo dedicam à etapa de levantamento de requisítos de software?

    \item Quais autores acordam boas práticas para um bom processo de elicitação, modelagem e análise de requisitos?

    \item De que maneira o método Design Sprint reduz o processo de design de um projeto para 5 dias?
    
    \item Agile practices in requirements management

    \item Elicitation process, modeling and requirements analysis

    \item Scope Management Process 

    \item Method of the Sprint design process
\end{itemize}

\subsubsection{Definição das Bases de Pesquisa}
Como bases de pesquisa, foram definidas as seguintes:

- CAPES

- IEEE

- WEB OF SCIENCE

- ACM

- SCOPUS

\subsubsection{Palavras-Chave}
Como etapa do processo de seleção de artigos, foi utilizado como método de seleção a busca pelas seguintes palavras-chave:

-Sprint Design

-Elicitation / Elicitação

-Requirements / Requisítos

-Agile practices / Práticas Ágeis

-Modeling / Modelagem

-Analysis / Análise

-Startups

-Modeling Techniques / Técnicas de Modelagem

-Project Performance / Desempenho do projeto

-Small scale projects / Projetos de pequena escala

\subsubsection{Procedimento de Seleção de Artigo}
Para realizar a seleção dos artigos a serem estudados, foram definidos critérios de inclusão
e exclusão de artigos, sendo necessário para evitar que artigos
indesejados sejam inclusos na pesquisa.}

\subsubsection{Avaliação da Qualidade}
A avaliação da qualidade do artigo deu-se por meio da análise de seu conteúdo, com foco principalmente no capítulo de introdução, resultados e conclusões. 
Sendo que para isso foram definidas perguntas a serem respondidas para validação da qualidade do artigo:

\begin{alin}

\item 1. O estudo apresentado é condizente com os objetivos da pesquisa? 
\item 2. As evidências apresentadas são relevantes para o desenvolvimento da abordagem?
\item 3. O artigo apresenta qualis igual ou superior a B1?
\item 4. O artigo se encontra em uma das categorias (dissertações, teses, conferências ou \textit{journal}?
\end{alin}
\subsubsection{Processo de Análise de Artigo}
\begin{align}
1. Busca na Base

2. Leitura do título com base nos critérios de seleção

3. Avaliação da disponibilidade de leitura do artigo

4. Leitura do resumo com base nos interesses pertinentes ao tema

5. Avaliação da qualidade do artigo com base na Avaliação da Qualidade

6. Leitura do artigo completo
\end{align}

\subsubsection{Processo de Inclusão e Exclusão}
\begin{align}
\item {Inclusão:}

- Artigos relacionados ao processo de gerênciamento de requisítos

- Artigos relacionados a aplicação de práticas ágeis 

- Artigos publicados a partir de 2004
\end{align}

\begin{align}
\item {Enclusão:}

- Artigos indisponíveis para leitura

- Artigos que possue outro idioma que não seja Português ou Inglês

- Artigos não relacionados a área de tecnologia
\end{align}



\hypertarget{A}{}
\section{\textbf{Condução da revisão:}}

\hypertarget{A}{}
\section{\textbf{Relato da revisão:}}







% \chapter{Segundo Apêndice}
% Texto do segundo apêndice.
% \end{apendicesenv}